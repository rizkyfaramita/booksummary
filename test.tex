\documentclass{article}

\title{Summary of Fundamentals of Information Systems, 8th  Edition}

\begin{document}
	\begin{center}
		\setcounter{page}{1}
		{\mdseries\Large Resume Bab 1}\\[4.5pt]
		{\mdseries\large Fundamentals of Information Systems, 8th  Edition}\\[1cm]
		\textbf{Rizky Faramita}\\[1pt]
		13515055@std.stei.itb.ac.id\\[1pt]
		Institut Teknologi Bandung\\[5pt]
	\end{center}
	\noindent
	\textwidth = 580pt

\section{Konsep Sistem Informasi}
Sebelum dapat memahami apa itu sistem informasi, kita harus mengetahui terlebih dahulu perbedaan dari data, informasi, dan pengetahuan. Data adalah fakta yang masih bersifat mentah, contohnya jumlah karyawan, total pesanan, dan lain-lain. Informasi adalah koleksi dari beberapa fakta yang tersusun dan telah diproses sehingga memiliki nilai tambah lebih. Pengetahuan adalah pemahaman dari sekumpulan informasi untuk dapat digunakan dalam menyelesaikan suatu tugas atau memberikan keputusan.\\

\noindent Sistem informasi merupakan sebuah set komponen yang dapat mengambil, memanipulasi, menyimpan, dan mengeluarkan data serta informasi. Pada dasarnya, sistem informasi dapat terbagi menjadi dua, yakni manual dan \emph{computerized}. Namun, pada mata kuliah ini, sistem informasi yang dimaksud adalah sistem informasi berbasis komputer atau CBIS. Infrastruktur CBIS terdiri dari beberapa komponen, mulai dari perangkat keras, perangkat lunak, basis data, telekomunikasi, jaringan, internet, sumber daya manusia, hingga prosedur.


\section{Sistem Informasi di Dunia Kerja}
Di dunia kerja, terdapat berbagai jenis sistem informasi, seperti \emph{Electronic Commerce, Mobile Commerce, Electronic Business, Enterprise System, \& Specialized Business Information System}. \emph{Enterprise System} sendiri dapat terbagi lagi menjadi empat macam, yaitu \emph{Transaction Processing Systems, Enterprise Resource Planning, Management Information Systems} dan \emph{Decision Support Systems}. Selain itu, \emph{Specialized Business Information Systems} juga memiliki beberapa contoh diantaranya adalah AI, robotik, \emph{neural network}, VR, dan lain-lain.\\

\noindent Berikut merupakan informasi penting terkait implementasi sistem informasi di dunia kerja.

\subsection{Perkembangan Sistem}
Seiring berkembangnya zaman, sistem informasi pun juga kian beradaptasi untuk menyeimbangi perubahan-perubahan tersebut. Terdapat dua mekanisme, yakni investigasi, analisis, dan desain \& konstruksi, integrasi dan pengujian, implementasi, operasi dan \emph{maintenance}, serta disposisi (penutupan).

\subsection{Pengaruh Sistem Informasi pada Organisasi}
\subsubsection{Inovasi}
Inovasi merupakan salah satu elemen penting dalam sebuah perusahaan untuk dapat menentukan keberlangsungannya. Namun, adopsi dari inovasi tersebut tidak terjadi dalam satu waktu pada seluruh pekerja. Tetapi, hal tersebut merupakan suatu proses yang dapat diterima cepat lambatnya bergantung dari penerima inovasi tersebut pula.
\subsubsection{Perubahan pada Organisasi}
Sistem informasi juga turut memengaruhi bagaimana sebuah organisasi baik profit maupun nonprofit untuk merencanakan dan mengimplementasikan perubahan.
\subsubsection{Competitive Advantage \& Financial Evaluation}
Competitive Advantage memiliki 5 faktor yaitu rivari, ancaman pemain baru, ancaman produk atau layanan pengganti, bargaining power dari pembeli, dan penjual. Financial Evaluation yang dapat diterapkan contohnya adalah payback period dan IRR.

\section{Isu Etika \& Sosial terkait Sistem Informasi}
\subsection{Tantangan Global}
Tantangan global yang mungkin dihadapi apabila terjun di dunia sistem informasi diantaranya adalah budaya, waktu, jarak, serta hukum wilaya, regional, dan nasional.
\subsection{Tipikal Jabatan Sistem Informasi}
Berikut beberapa profesi yang berkaitan erat dengan sistem informasi: \emph{Chief Information Officer, Senior IS Managers, Operation Roles, Development Roles, Support, Certification}.

\end{document}

